\documentclass[oneside]{book}

\usepackage{graphics}
\usepackage{wrapfig}
\usepackage{rotating}
\usepackage{gji}             % reference style
\usepackage{small_caption}   % caption font
\input{LGRenc.def}           % lower case sans serif Greek
%% 
%% This is file `LGRcmss.fd',
%% generated with the docstrip utility.
%% 
%% The original source files were:
%% 
%% greek.fdd  (with options: `fd,LGRcmss')
%% 
%% This is a generated file. 
%% 
%% Copyright 1996 Johannes L. Braams and any individual authors 
%% listed elsewhere in this file. All rights reserved. 
%% 
%% For further copyright information, and conditions for modification 
%% and distribution, see any other copyright notices in this file. 
%% 
%% This file is part of the Babel system, release 3.6. 
%% --------------------------------------------------- 
%% This system is distributed in the hope that it will be useful, 
%% but WITHOUT ANY WARRANTY; without even the implied warranty of 
%% MERCHANTABILITY or FITNESS FOR A PARTICULAR PURPOSE. 
%% 
%% For error reports concerning UNCHANGED versions of this file no 
%% more than one year old, see bugs.txt. 
%% 
%% Please do not request updates from me directly. Primary 
%% distribution is through the CTAN archives. 
%% 
%% 
%% IMPORTANT COPYRIGHT NOTICE: 
%% 
%% You are NOT ALLOWED to distribute this file alone. 
%% 
%% You are allowed to distribute this file under the condition that it 
%% is distributed together with all the files listed in manifest.txt. 
%% 
%% If you receive only some of these files from someone, complain! 
%% 
%% 
%% Permission is granted to customize the declarations in this file to 
%% serve the needs of your installation, provided that you comply with 
%% the conditions in the file legal.txt. 
%% 
%% However, NO PERMISSION is granted to distribute a modified version 
%% of this file under its original name. 
%% 
%% 
%% MODIFICATION ADVICE: 
%% 
%% If you want to customize this file, it is best to make a copy of 
%% the source file(s) from which it was produced. Use a different 
%% name for your copy(ies) and modify the copy(ies); this will ensure 
%% that your modifications do not get overwritten when you install a 
%% new release of the standard system. 
%% 
%% You can then easily distribute your modifications by distributing 
%% the modified and renamed copy of the source file, together with a 
%% suitable .ins file, taking care to observe the conditions in 
%% legal.txt; this will ensure that other users can safely use 
%% your modifications. 
%% 
%% The names of the source files used are shown above. 
%% 
%% 
%% 
\ProvidesFile{LGRcmss.fd}
        [1997/01/09 v1.0c
  Greek Sans Serif]
\DeclareFontFamily{LGR}{cmss}{}
\DeclareFontShape{LGR}{cmss}{m}{n}
    {
      <6>     ma55g6
      <7>     ma55g7
      <8>     ma55g8
      <9>     ma55g9
      <10>    ma55g10
      <10.95> ma55g109
      <12>    ma55g12
      <14.4>  ma55g144
      <17.28> ma55g172
      <20.74> ma55g207
      <24.88> ma55g248}{}
\DeclareFontShape{LGR}{cmss}{m}{it}{ <-> ssub * cmss/m/n}{}
\DeclareFontShape{LGR}{cmss}{m}{sl}{ <-> ssub * cmss/m/n}{}
\DeclareFontShape{LGR}{cmss}{m}{sc}{ <-> ssub * cmss/m/n}{}
\DeclareFontShape{LGR}{cmss}{b}{n}{  <-> ssub * cmss/m/n}{}
\DeclareFontShape{LGR}{cmss}{b}{it}{ <-> ssub * cmss/m/n}{}
\DeclareFontShape{LGR}{cmss}{b}{sl}{ <-> ssub * cmss/m/n}{}
\DeclareFontShape{LGR}{cmss}{b}{sc}{ <-> ssub * cmss/m/n}{}
\DeclareFontShape{LGR}{cmss}{bx}{n}{  <-> ssub * cmss/m/n}{}
\DeclareFontShape{LGR}{cmss}{bx}{it}{ <-> ssub * cmss/m/n}{}
\DeclareFontShape{LGR}{cmss}{bx}{sl}{ <-> ssub * cmss/m/n}{}
\DeclareFontShape{LGR}{cmss}{bx}{sc}{ <-> ssub * cmss/m/n}{}
\endinput
%% 
%% End of file `LGRcmss.fd'.
           % lower case sans serif Greek

%%%%%%%%%%%%%%%%%%%%%%%%%%%%%%%%%%%%%%%%%%%%
                                           %
% Trying to match Tony's textures setup    %
                                           %
\setlength{\textwidth}{4.5in}              %
\setlength\textheight{42\baselineskip}     %
\setlength\topmargin{.75in}                %
\setlength\oddsidemargin{1.0in}            %
\setlength\evensidemargin{1.0in}           %
\setlength\marginparwidth{.75in}           %
                                           %
%%%%%%%%%%%%%%%%%%%%%%%%%%%%%%%%%%%%%%%%%%%%

\input{newcommands}

\pagestyle{empty}

\begin{document}

%\setcounter{chapter}{5}
%\setcounter{figure}{7}
%\begin{sidewaysfigure}
%\centering
%\rotatebox{270}
%{
%\includegraphics{../figures/chap05/fig08.eps}
%}
%\caption[global seismicity]{
%Epicentral locations and source mechanisms of 10,219 earthquakes
%in the Harvard CMT catalogue, with focal depths less than~50~km,
%during the period 1976--1997.  The size of each beachball
%is proportional to the logarithm of the seismic moment $M_0$.
%The world map is a cylindrical equal-area projection, with
%landmasses shaded.  (Courtesy of E.\ Larson.)
%}
%\end{sidewaysfigure}

\setcounter{chapter}{8}
\setcounter{figure}{8}
\begin{sidewaysfigure}
\centering
\rotatebox{270}
{
\includegraphics{../figures/chap08/fig09.eps}
}
\caption[sphmodefreqs]{
Dispersion diagram showing the spheroidal
oscillations of the isotropic PREM model.
All of the predominantly elastic eigenfrequencies
${}_n\omega_l^{\rm S}/2\pi$ below 20 mHz are plotted
versus the spherical-harmonic degree~$l$.
Small numbers on left label the radial-mode
($n=0$) eigenfrequencies.
See Figure~8.10 for greater detail.
}
\end{sidewaysfigure}

\setcounter{chapter}{8}
\setcounter{figure}{9}
\begin{sidewaysfigure}
\rotatebox{270}
{
\centering
\includegraphics{../figures/chap08/fig10.eps}
}
\caption[sphmodefreqs2]{
Detailed view of the lower left corner of the
spheroidal-mode dispersion diagram in Figure~8.9.
Leftmost numbers 0 through 16 label
the radial-mode ($n=0$) eigenfrequencies;
repeated numbers 0 through 39 label
the low-frequency and high-frequency
ends of the various overtone branches.
}
\end{sidewaysfigure}

\setcounter{chapter}{8}
\setcounter{figure}{11}
\begin{sidewaysfigure}
\centering
\rotatebox{270}
{
\includegraphics{../figures/chap08/fig12.eps}
}
\caption[threemodes]{
Subdivision of the high-frequency, high-overtone spheroidal
oscillations into mantle ${\rm ScS}_{\rm SV}$ modes ({\em left\/}),
PKIKP modes ({\em middle\/}) and inner-core ${\rm J}_{\rm SV}$
modes ({\em right\/}). The ${\rm ScS}_{\rm SV}$ modes have $f_{\mu}>0.5$
and $f_{\mu,\mbox{\scriptsize\,inner core}}\ll f_{\mu,\mbox{\scriptsize\,mantle}}$;
the PKIKP modes have $f_{\kappa}>0.5$;
the ${\rm J}_{\rm SV}$ modes have $f_{\mu}>0.5$ and
$f_{\mu,\mbox{\scriptsize\,inner core}}\gg f_{\mu,\mbox{\scriptsize\,mantle}}$.
}
\end{sidewaysfigure}

\setcounter{chapter}{9}
\setcounter{figure}{6}
\begin{sidewaysfigure}
\centering
\rotatebox{270}
{
\scalebox{0.95}{
\includegraphics{../figures/chap09/fig07.eps}
}
}
\caption[ScS&J&PKIKPkernels]{
Fr\'{e}chet kernels $K_{\alpha}$ ({\em dotted line\/}),
$K_{\beta}$ ({\em dashed line\/}) and $K_{\rho}^{\prime}$
({\em solid line\/}) for a number of ${\rm ScS}_{\rm SV}$
modes ({\em top row\/}), PKIKP modes ({\em middle row\/}) and
${\rm J}_{\rm SV}$ modes ({\em bottom row\/}) of degree $l=2$.
The displacements $U$ and $V$ are displayed in
Figure~8.14.  Each graph is scaled independently,
so that the maximum values of the kernels are the same.
}
\end{sidewaysfigure}

\setcounter{chapter}{12}
\setcounter{figure}{14}
\begin{sidewaysfigure}
\centering
\rotatebox{270}
{
\includegraphics{../figures/chap12/fig15.eps}
}
\caption[Spheroidal Dispersion]{
Spheroidal dispersion diagrams for the crustless
version of model 1066A.
({\em Left\/}) Exact eigenfrequencies calculated
by numerical integration.
({\em Right\/}) Asymptotic eigenfreqencies
in P-SV Regimes II--X.
See Figure~12.16 for
a magnified view of the upper left corner.
(Courtesy of L. Zhao.)
}
\end{sidewaysfigure}

\setcounter{chapter}{14}
\setcounter{figure}{23}
\begin{sidewaysfigure}
\begin{center}
\rotatebox{270}
{
\includegraphics{../figures/chap14/fig24.eps}
}
\end{center}
\caption[pas_spec]{
Radial-component amplitude spectra
%$|\hat{\bf r}\cdot{\bf a}({\bf x},\omega)|$
of the June 9, 1994 Bolivia
deep-focus earthquake recorded at station PAS in
Pasadena, California.  Solid line shows observed
spectrum; dashed line shows synthetic coupled-mode
spectrum.  Rotation, ellipticity and the lateral variations
in shear-wave speed $\delta\hspace{-0.2 mm}\beta$ of model
SKS12WM13 have been accounted for in the calculations.
The vertical arrays of mode labels identify the twenty-two
super-multiplets considered.  A Hann taper has been applied
to both 35-hour time series prior to Fourier transformation.
}
\end{sidewaysfigure}

\setcounter{chapter}{14}
\setcounter{figure}{24}
\begin{sidewaysfigure}
\begin{center}
\rotatebox{270}
{
\includegraphics{../figures/chap14/fig25.eps}
}
\end{center}
\caption[ptga_spec]{
Radial-component amplitude spectra
%$|\hat{\bf r}\cdot{\bf a}({\bf x},\omega)|$
of the February 17, 1996 Irian
Jaya earthquake recorded at station PTGA in Pitinga,
Brazil.   Solid line shows observed
spectrum; dashed line shows synthetic coupled-mode
spectrum.  Rotation, ellipticity and the lateral variations
in shear-wave speed $\delta\hspace{-0.2 mm}\beta$ of model
SKS12WM13 have been accounted for in the calculations.
The vertical arrays of mode labels identify the twenty-two
super-multiplets considered.  A Hann taper has been applied
to both 35-hour time series prior to Fourier transformation.
}
\end{sidewaysfigure}

\setcounter{chapter}{15}
\setcounter{figure}{11}
\begin{sidewaysfigure}
\begin{center}
\rotatebox{270}
{
\includegraphics{../figures/chap15/fig12.eps}
}
\end{center}
\caption[liu&tromp Fig 9]{
Comparison of the exact ({\em long dashed line\/}) and
perturbation-theoretical ({\em short dashed line\/}) ray paths
in model SKS12WM13. The unperturbed ray path
in PREM is also shown ({\em solid line\/}).  The source
is situated on the equator and Greenwich Meridian, and the
receiver is situated due east at an epicentral distance
$\Theta=75^{\circ}$.  ({\em Top\/}) P, PcP, S and ScS
rays projected onto a cross-section of the source-receiver
great-circle plane. Shading depicts the relative wave-speed
perturbations $-0.8\%\leq\delta\hspace{-0.1 mm}\alpha/\alpha\leq 0.8\%$
and $-1.5\%\leq\delta\hspace{-0.2 mm}\beta/\beta\leq 1.5\%$.
({\em Bottom\/}) Projection of the same ray paths onto the
surface of the Earth.
}
\end{sidewaysfigure}

\setcounter{chapter}{16}
\setcounter{figure}{17}
\begin{sidewaysfigure}
\begin{center}
\rotatebox{270}
{
\includegraphics{../figures/chap16/fig19.eps}
}
\end{center}
\caption[phase_mapsL]{
Global Love-wave phase-speed maps $\delta c/c$
(in percent) at periods of 35, 50, 75 and 150~seconds.
Notice that the grey scales differ from map to map.
}
\end{sidewaysfigure}

\setcounter{chapter}{16}
\setcounter{figure}{18}
\begin{sidewaysfigure}
\begin{center}
\rotatebox{270}
{
\includegraphics{../figures/chap16/fig20.eps}
}
\end{center}
\caption[phase_mapsR]{
Global Rayleigh-wave phase-speed maps $\delta c/c$
(in percent) at periods of 35, 50, 75 and 150~s.
Notice that the grey scales differ from map to map.
}
\end{sidewaysfigure}

\setcounter{chapter}{16}
\setcounter{figure}{19}
\begin{sidewaysfigure}
\begin{center}
\rotatebox{270}
{
\includegraphics{../figures/chap16/fig07.eps}
}
\end{center}
\caption[crustal]{
Relative perturbation $\delta c/c$ in the phase speed
of fundamental-mode Love waves ({\em left\/}) and
Rayleigh waves ({\em right\/}) due to lateral variations
in the thickness and structure of the Earth's crust.
The perturbations are shown at two periods: 35~s
({\em top\/}) and 75~s ({\em bottom\/}). Note that
the grey scales differ from map to map.
}
\end{sidewaysfigure}

\end{document}
