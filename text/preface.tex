\setcounter{chapter}{0}
\chapter*{Preface}

\thispagestyle{plain}
Five years ago, in the summer of 1993, we set out to
write a slender monograph tentatively entitled
{\em The Free Oscillations of the Earth\/}.
With each e-mail exchange of draft chapters,
our modest ambitions mushroomed; the final result is this
book---{\em Theoretical Global Seismology\/}---an
advanced treatise intended to be read by graduate
students and researchers in geophysics and allied fields.
Although the present title is more indicative of the scope
than the original, the contents nevertheless reflect the book's origins.
We devote our attention almost exclusively to the forward
problem of computing synthetic seismograms upon a realistic
three-dimensional model of the Earth, with a strong emphasis
on the normal-mode summation method.  Free oscillations have
many close associations with surface waves,
and we consider them in some detail as well; we give
shorter shrift to body waves, and do not discuss seismic
instrumentation, data analysis procedures, or geophysical
inverse theory at all.

The introductory chapter recounts the history of free-oscillation
and surface-wave research, beginning with the earliest theoretical
investigation of the oscillations of an elastic sphere in the 1820's,
through the first observation of the gravest terrestrial oscillations
following the great Chile earthquake of 1960, and concluding with
the initial determinations of global upper-mantle heterogeneity
using digitally recorded seismograms in the 1980's.
The remainder of the text---like ancient Gaul---is
divided into three parts. 
In Part I---{\em Foundations\/}---we derive the
linearized equations of motion
governing both an elastic and anelastic Earth
subject to a non-hydrostatic
state of initial stress, and show how to express the elastic-gravitational
response to an arbitrary earthquake source as a sum of free oscillations
or normal modes.  We conclude with a discussion of the Rayleigh-Ritz
method, which yields a truncated matrix formulation that is identical to the
classical theory of the small oscillations of a system with a finite
number of degrees of freedom, generalized to account for rotation
and anelasticity.  In Part II---{\em The Spherical Earth\/}---we restrict
attention to the case of an Earth model that is non-rotating and spherically
symmetric; the toroidal and spheroidal eigenfrequencies and eigenfunctions of
such a model can be found essentially exactly by numerical integration
of the governing radial differential equations.  We show how to calculate
synthetic seismograms on a spherical Earth by means of normal-mode summation,
and discuss the propagation of Love and Rayleigh surface waves as well as
mode-ray duality.  These results form the basis for the more general
considerations in Part III---{\em The Aspherical Earth\/}---where we use
perturbation theory to treat the splitting and coupling of the normal-mode
multiplets produced by the Earth's rotation, ellipticity and other departures
from spherical symmetry, and JWKB theory to describe the propagation of both
body waves and surface waves upon a laterally heterogeneous Earth.

\thispagestyle{myheadings} \markboth{PREFACE}{PREFACE}
The three parts are arranged in order of decreasing ``shelf life''.
The fundamental equations and results obtained in
Part I are applicable to a very general Earth model, and should provide
the basis for discussions of the elastic-gravitational deformation
of the Earth into the foreseeable future.  The results pertaining to a
spherical Earth in Part II are likewise well established;
only relatively minor numerical details are likely to change as the
spherically averaged structure of the Earth continues to be refined.
The approximate methods of dealing with the Earth's lateral heterogeneity
which we discuss in Part III are not as well developed;
three-dimensional global tomography is an extremely active research field
at the present time, and improvements in the procedures and results
described here seem likely in the future.
In addition to the fifteen chapters in Parts I through III, there are four
mathematical appendixes devoted to vectors and tensors, ordinary and
generalized spherical harmonics, and the matrix machinery needed to
calculate coupled-mode synthetic seismograms on a rotating, anelastic,
laterally heterogeneous Earth.
 
Variational principles appear in a number of guises, and provide
a unifying thread which serves to knit the various chapters together. 
We enunciate Hamilton's principle for a general elastic Earth
model in Chapter~3, and discuss its frequency-domain analogue,
Rayleigh's principle, on both a non-rotating and rotating
Earth in Chapter~4.  We extend Rayleigh's principle to an
anelastic Earth in Chapter~6, deduce the equivalent
elastic and anelastic matrix
principles in Chapter~7, and utilize the orthonormality of the
surface spherical harmonics to obtain a purely radial variational
principle on a spherically symmetric Earth in Chapter~8.
The one-dimensional and three-dimensional versions
of Rayleigh's principle provide the basis for the
spherical and aspherical perturbation
analyses in Chapters~9 and~13.  Finally, we develop ray theory
for body waves and JWKB theory for surface waves on a smooth
laterally heterogeneous Earth using an associated slow variational
principle in Chapters~15 and~16.

The subject matter of this book may be described as mathematical
in the sense that there is a high proportion of equations to words;
however, all of the theoretical considerations are purely formal,
with no attempt at rigor whatsoever.  We are not finicky about
the continuity and differentiability of displacement, strain and
stress fields, or the open or closed nature of regions within
the Earth, except where it matters to get the physics right.
The only mathematical property of the elastic-gravitational
operator governing the Earth's free oscillations which is
considered to be physically significant is whether or not
it is Hermitian.  We blithely assume that the normal modes
of an elastic Earth model are complete, ignore the presence
of a branch cut in developing a mode-sum representation of
the response of an anelastic Earth, manipulate infinite
matrices without regard for convergence, and seldom worry
about the precise nature of the equality in spherical-harmonic
and other infinite orthonormal eigenfunction expansions.

Sections denoted by a star${}^{\textstyle{\star}}$
contain more esoteric material which may be omitted upon a first
reading.  Many of the starred sections deal with the theoretical
complications introduced by the Earth's rotation; for example,
in analyzing the influence of anelasticity upon the free
oscillations, it is necessary to introduce the
dual eigenfunctions $\overline{\bs}$
associated with the ``anti-Earth'' having the opposite sense
of rotation, as well as the ordinary eigenfunctions $\bs$
of the Earth itself.  A few unstarred sections make use of these
dual eigenfunctions in the interest of brevity and maximum generality;
uninterested readers may simply eliminate the overbars, since in the
absence of rotation the eigenfunctions and their duals
coincide: $\overline{\bs}=\bs$.

\thispagestyle{myheadings} \markboth{PREFACE}{PREFACE}
We are deeply indebted to many colleagues for their generous
support and assistance during the preparation of this book.
First and foremost, we would like to thank Freeman Gilbert
for his barrage of encouraging e-mails, filled with valuable
commentary upon a variety of topics---ranging from the inherent
positivity of the group speed to the application of ray theory
to stealth-aircraft detection.  We also wish to express our
sincere gratitude to Guust Nolet, whose detailed and constructive
criticism, particularly of the appendixes, was extremely helpful.
An early, incomplete draft was reviewed by Brian Kennett,
Guy Masters and Barbara Romanowicz; they suggested
a number of improvements which have been incorporated in the
final version.  Several people graciously complied with our
request to read a particular chapter
devoted to their field of expertise; we especially wish
to thank Henk Marquering and Roel Snieder for their
remarks regarding Chapter~11, Li Zhao for his careful
review of Chapter~12, and Colin Thomson
for his advice on Chapter~15.  Appendices~B and~C on ordinary
and generalized spherical harmonics are based in part upon lecture
notes by George Backus and John Woodhouse.  Further suggestions
for improvements and additions were provided by Chris Chapman, Adam Dziewonski,
Andy Jackson, Paul Richards, and Philippe Lognonn\'{e}.  Our
foray into the early German surface-wave literature was aided
by Thomas Meier.  Finally, we would like to acknowledge our
indebtedness to Miaki Ishii, whose thorough review of the entire
manuscript is very much appreciated.

We are grateful to many individuals for helping us
to assemble the more than 225 illustrations.  The theoretical
spectra and seismograms and eigenfunction and Fr\'{e}chet kernel plots
were almost all produced by students in Jeroen Tromp's global seismology
courses at Harvard and MIT.  We thank John He, Yu Gu, Rishi Jha,
Hrafnkell K\'{a}rason, Erik Larson, Xian-Feng Liu, Jeff McGuire, Meredith Nettles,
Frederik Simons and Mark Taylor for their help in this endeavor.
A number of colleagues, including G\"{o}ran Ekstr\"{o}m, Guy Masters,
Joe Resovsky, Mike Ritzwoller, Barbara Romanowicz, Genevieve Roult,
Peter Shearer, Zheng Wang, Shingo Watada, Ruedi Widmer and Li Zhao,
provided us with additional figures; we sincerely thank them all.
Most of the cartoons were ably drafted from our slapdash sketches
by Dearbhla McHenry and Leslie Hsu; the indispensable Leslie also
organized, re-sized, touched-up and unified all of the figures for
encapsulation into the final camera-ready copy. 

The labor of composing, formatting and typesetting this behemoth
of more than 1000 pages and 3800
numbered equations was ameliorated by
{L\kern-.36em\raise.3ex\hbox{\sc a}\kern-.15em
T\kern-.1667em\lower.7ex\hbox{E}\kern-.125emX},
{B\kern-.05em{\sc i\kern-.025em b}\kern-.08em
T\kern-.1667em\lower.7ex\hbox{E}\kern-.125emX}
and {\em MakeIndex\/}; we benefitted from the
expertise of Bob Fischer and Erik Larson.
Meredith Nettles and Yu Gu indulged our paranoia by religiously
backing up all of the chapter and figure files.  It has been a pleasure working
with the capable staff at Princeton University Press,
particularly Jack Repcheck, who has guided this book
to publication from the outset, and Jennifer Slater,
who did a splendid job of copy editing.

The awards of a John Simon Guggenheim Memorial Foundation
Fellowship to Tony Dahlen and a David and Lucile Packard
Foundation Fellowship to Jeroen Tromp are greatly appreciated.
In addition, Tony Dahlen would like to express his sincere thanks
to Raul Madariaga, Jean-Paul Montagner and Philippe Lognonn\'{e}
for their support and gracious hospitality during his
1993-1994 sabbatical leave at the Institut de Physique du Globe
de Paris.  A preliminary draft of Part~I was completed and
the remainder of the book was outlined during this visit.
Further financial support was provided by grants from the
National Science Foundation to the two authors at Princeton
and Harvard.

Finally, we would like to thank Elisabeth, Tracey and Alex for
patiently putting up with our seismological gibberish and preoccupation
with this project during the past five years.  The fact of the matter is
that we should be grateful that they put up with us at all.
\\ \\
Princeton and Cambridge
\\
June 1998

\thispagestyle{myheadings} \markboth{PREFACE}{PREFACE}
