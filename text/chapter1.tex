\chapter{Historical Introduction}
\pagenumbering{arabic}
\setcounter{page}{3}

After every major earthquake, the Earth rings like a large bell
for several days.  These free oscillations of the Earth
are routinely detected at modern broad-band seismographic stations,
which are now distributed globally. The eigenfrequencies and decay
rates of the vibrations can be measured and used to constrain the
radial and lateral distribution of density, seismic wave speed and
anelastic attenuation within the interior.  The observed amplitudes and phases
can likewise be used to infer the origin times, hypocentral locations,
seismic moments, and fault geometries of the earthquakes responsible
for the excitation.
The analysis of the free oscillations of the Earth and the allied
normal-mode methods employed in the determination of the Earth's internal
structure and the source mechanisms of earthquakes---the topics considered
in this book---constitute one of the cornerstones of quantitative seismology.
Excellent reviews which summarize the state of progress at two pivotal
points in the development of the field are provided by
\textcite{stoneley61}, \textcite{lapwood&usami81} and \textcite{buland81}.
This introduction contains our own brief historical
survey of research on terrestrial free oscillations
and the associated propagating surface waves,
focusing upon the
theoretical and observational advances made prior to 1985.
More recent developments are described---with little attention
to their historical context---in subsequent chapters of the book.

\section{Early Theoretical Studies}

The theoretical analysis of the Earth's normal modes was
initiated over one and one-half centuries ago by the French
mathematician Poisson.  In a remarkable memoir presented
to the Paris Academy of Sciences in August 1828, he developed
a general theory of deformation for solid materials, based upon
``la consid\'{e}ration des actions mutuelles de leurs mol\'{e}cules'',
and applied it to a large number of special elastostatic
and elastodynamic problems, including the determination
of the frequencies of the purely radial oscillations of a
homogeneous, non-gravitating sphere (Poisson \citeyear{poisson29}).
These investigations, together with the work of his contemporaries Navier and
Cauchy, laid the foundations for the modern theory of linear elasticity.
The equations of equilibrium and vibration derived by Poisson are now
recognized to be incomplete, inasmuch as they characterize the elastic
response of an isotropic solid in terms of a single elastic parameter rather
than two; the radial-mode eigenfrequencies and eigenfunctions he obtained are,
however, correct in the special case that we now refer to as a Poisson solid,
\index{Poisson solid}%
which has $\kappa=\fivethirds\mu$, where $\kappa$ is the incompressibility
and $\mu$ is the rigidity.  Not being a physicist or natural philosopher,
Poisson did not seek to estimate or calculate the numerical free
periods of radial vibration of the Earth or any man-made spherical objects,
but rather expressed his final results in terms of dimensionless ratios.

The first numerical estimate of a vibrational eigenfrequency of the
Earth was made by Lord Kelvin in 1863.  The prevailing opinion of most
geologists and geophysicists at the time was that the Earth was completely
molten, except for a thin crust of solid rock.  Supporting evidence for
this conclusion included the good agreement of the observed ellipticity
of figure with the hydrostatic theory of Clairaut, the rapid increase of
temperature with depth in mines, and the eruption of lava from active volcanoes.
Seeking to challenge this view, Kelvin calculated the fundamental degree-two
spheroidal-mode eigenfrequency of the Earth using two different assumptions
(Thomson \citeyear{kelvin63a}).  For a self-gravitating fluid Earth he found
the period of this mode---now designated ${}_0{\rm S}_2$---to be 94 minutes,
whereas for a solid Earth having the same rigidity as steel he asserted that
the period would be approximately 69 minutes.
The first value was obtained by means of an exact dynamical
analysis for a homogeneous, incompressible ($\kappa=\infty$)
fluid ($\mu=0$) sphere (Thomson \citeyear{kelvin63b}),
whereas the second was estimated on the basis
of the time required for a shear wave to transit the diameter,
using a laboratory value for the rigidity of steel obtained from
his brother James in Glasgow.
Lacking a means to measure the terrestrial eigenfrequencies, Kelvin
devised an ingenious procedure for determining the mean rigidity of the Earth
based upon the height of the fortnightly and monthly tides.
He noted that the gravitational attraction of the Moon and Sun
must raise bodily tides within the solid Earth as well as the tides
within the oceans familiar to all seafarers, and pointed out that the
observed oceanic tides, which are measured with respect to the deformed
seafloor, should be nearly zero on a molten Earth.
He determined the elastic-gravitational response of a homogeneous,
incompressible solid sphere to an applied tidal potential,
and showed that the oceanic tides on an elastic Earth should be reduced
relative to their equilibrium value on a rigid Earth by an amount
$\eta=(19\mu/2\rho ga)(1+19\mu/2\rho ga)^{-1}$,
where $\rho$ is the density, $a$ is the radius, and $g$
is the surficial acceleration of gravity.  Since Kelvin's
analysis was quasi-static, this elastic-Earth reduction factor
could not be applied directly to the dominant semi-diurnal and
diurnal tides; however, he argued that it should be applicable
to the fortnightly and monthly tides, since they are largely
devoid of ocean-basin resonance.  The available fortnightly
and monthly observations were insufficiently accurate for
his purpose; accordingly, he persuaded the British Association
to establish a Tidal Committee charged with ``the evaluation
of the long-period tides for the purpose of answering the question
of the Earth's rigidity''.  The harmonic analysis of 66 years
of tidal observations from fourteen British, French and Indian ports was
undertaken by George Darwin, who published his results in the
second edition of the {\em Treatise on Natural Philosophy\/}
(Thomson \& Tait \citeyear{thomson&tait83}).
Averaging the results from all ports
and both tides, Darwin found that $\eta=0.676 \pm 0.076$, 
indicating that the tidal-effective rigidity of the Earth
is indeed ``about equal to that of steel''.
\index{tidal-effective rigidity}%
\index{rigidity!tidal-effective}%
This celebrated
conclusion corroborated Kelvin's 69-minute estimate of the period of
the ${}_0{\rm S}_2$ mode, grounding it upon a measured physical
property of the Earth.

An early theoretical investigation of the toroidal modes of
a homogeneous sphere was undertaken by \textcite{jaerisch80};
however, the first comprehensive treatment of the free oscillations
of a non-gravitating sphere is the classic analysis of \textcite{lamb82}.
He distinguished clearly between the spheroidal oscillations, which
he called ``vibrations of the first class'', and the toroidal oscillations,
which he called ``vibrations of the second class'', and concluded that
the period of the ${}_0{\rm S}_2$ mode for a steel sphere the
size of the Earth should be 65 minutes in the case $\kappa=\infty$
and 66 minutes in the case $\kappa=\fivethirds\mu$.  The good agreement
with Kelvin's order-of-magnitude estimate is to some
extent coincidental, since Lamb used
an improved, slightly higher value for the rigidity $\mu$ of steel;
the insensitivity to the value of the incompressibility $\kappa$ is a
consequence of the fact that the ellipsoidal deformation is dominated by shear.
Lamb conducted his analysis in terms of three-dimensional
Cartesian coordinates; however, it was
subsequently shown by \textcite{chree89} that the same results could be
obtained much more economically using spherical polar coordinates.
Such a spherical-harmonic representation of the elastic-gravitational
deformation of the Earth has been employed in the majority of theoretical
analyses ever since.

The proximity of the two rigorously derived periods---94 minutes for
a fluid sphere whose only restoring force is the mutual gravitation
of its parts and 65 minutes for a Poisson-solid sphere devoid of gravitational
attraction---is an indication that elasticity and self-gravitation must
play roughly equal roles in the ${}_0{\rm S}_2$ mode of the actual Earth.
\textcite{bromwich98} considered the free oscillations of an incompressible
solid sphere, and found that self-gravitation would reduce
the period from 65 to 55 minutes.  The correct treatment of a compressible,
self-gravitating sphere proved to be much more difficult, with many false
steps along the way during the next decade.  \textcite{jeans03} was the
first person to appreciate and point out the complications, but in order
to obtain a self-consistent system of governing equations, he found it
necessary to ``artificially annul gravitation in the equilibrium configuration,
so that this equilibrium configuration may be completely unstressed, and
each element of matter be in its normal state''.  His language illustrates
the problem that confounded these early workers: the classical linear theory
of elasticity was formulated to deal with deformations away from an unstressed,
unstrained equilibrium configuration; however, the total stress within a
self-gravitating body such as the Earth was clearly far too great to
be related to an infinitesimal strain by Hooke's law.  A generalization
of the classical theory, which decomposed the total stress
into a large {\em initial stress\/}
\index{initial stress}%
\index{stress!initial}%
balanced by the self-gravitation and an infinitesimal {\em incremental stress\/}
\index{incremental stress}%
\index{stress!incremental}%
calculable using Hooke's law, was proposed by Lord
Rayleigh (\citeyear{rayleigh06}):
\begin{quote}
``It appears to me that a satisfactory treatment of these problems must start
from the condition of the Earth as actually stressed by its self-gravitation
and that the difficulties to be
faced in following such a course may not be as
great as has been supposed\ldots.
The conclusion that I draw is that the
usual equations may be applied to matter in a state of stress, provided
that we allow for altered values of the elasticities.''
\end{quote}
The elastic-gravitational equations obtained by Rayleigh were not
correct, but his perspective upon the problem was highly influential.
\textcite{love07} elaborated upon Rayleigh's idea, and derived an
alternative system of equations; however, these too were incorrect
because he failed to distinguish between the Eulerian incremental
stress at a fixed point in space and the Lagrangian incremental stress
experienced by an observer attached to a material particle.  He soon
realized his error, and not only derived but solved the correct equations
in his monumental Adams Prize Essay {\em Some Problems of Geodynamics\/}
four years later (Love \citeyear{love11}).  His argument regarding the stress is
extremely clear and well worth repeating:
\begin{quote}
``The Earth ought to be regarded as a body in a state of {\em initial stress\/};
this initial stress may be regarded as a hydrostatic pressure balancing the
self-gravitation of the body in the initial state; the stress in the body,
when disturbed, may be taken to consist of the initial stress compounded
with an {\em additional stress\/}; the additional stress may be taken to be
connected with the strain, measured from the initial state as unstrained
state, by the same formulae as hold in an isotropic elastic body slightly
strained from a state of initial stress.  The theory, as here described,
is ambiguous in the following sense:---The initial stress at a point of the
body which is at $(x,\,y,\,z)$ in the strained state may be (1) the pressure
at $(x,\,y,\,z)$ in the initial state, or it may be (2) the pressure in the
initial state at that point which is displaced to $(x,\,y,\,z)$ when the body
is strained.  There can be little doubt that the second alternative is the
correct one.  A small element of the body is moved from one place to another,
and during the displacement it suffers compression and distortion.  It ought
to be regarded as carrying its initial pressure with it, and acquiring an
additional state of stress depending upon the compression and distortion.''
\end{quote}
Love found the period of the ${}_0{\rm S}_2$ mode of a
self-gravitating Poisson-solid sphere with the rigidity of steel
to be ``almost exactly 60 minutes''.  His results are strictly
applicable only to a {\em homogeneous\/} elastic, self-gravitating
\index{homogeneous sphere}%
sphere; however, the correct dynamical equations and boundary conditions
governing a sphere with {\em radially variable\/} properties $\kappa$,
$\mu$, $\rho$ are implicit in his analysis. 
The general equations do not appear
to have been written down explicitly until the work of \textcite{hoskins20}.
An apparently independent derivation is given by Jeffreys
in the first edition of {\em The Earth\/} (\citeyear{jeffreys24}).
Mindful of the checkered history of the subject, he presents
his analysis in a separate chapter, which is headed
by the celebrated inscription over the Gates of Hell from
Dante's {\em Inferno\/}, ``Lasciate ogni speranza, voi ch'entrate''.

Much of the work just summarized was motivated by cosmogonical
considerations and a desire to understand the instability
mechanisms of massive self-gravitating
configurations, with possible applications to the asymmetrical
distribution of continents and oceans and the origin by fission of
the Moon.  \textcite{jeans23} was the first person to place the normal modes
of the Earth in a seismological context; he showed that the superposition
of free oscillations
or standing waves excited by an earthquake source could alternatively
be regarded as a superposition of travelling body and surface waves.
His asymptotic relation $\om p=l+\half$ between the angular frequency $\om$
and spherical-harmonic degree $l$ of a normal mode of oscillation and
the seismological ray parameter $p$ of the corresponding wave
remains the basis for discussions of {\em mode-ray duality\/} to this day.
\index{mode-ray duality}%
\index{duality!mode-ray}%

The dynamical laws governing the free oscillations of the Earth can
alternatively be expressed in the form of a variational principle,
known as {\em Hamilton's principle\/}
\index{Hamilton's principle}%
in the time domain and as {\em Rayleigh's principle\/}
\index{Rayleigh's principle}%
in the frequency
domain.  Such principles can be established either by straightforward
manipulation of the linearized conservation laws and constitutive relation, or
from first principles by consideration of the kinetic and elastic-gravitational
potential energies accompanying an infinitesimal deformation.  An early
variational calculation of the quasi-static tidal response of an elastic,
self-gravitating Earth was attempted by \textcite{stoneley26b}; however, his
formulation of Rayleigh's principle, which led him to conclude ``that when the
expression for the total energy is evaluated, the gravitational terms are of
the order of 1/20 of the elastic terms'', is incorrect.  The first correct
variational calculations of the elastic-gravitational eigenfrequencies
of a realistic Earth model were made independently and more
or less simultaneously by Jobert (\citeyear{jobert56}; \citeyear{jobert57};
\citeyear{jobert61}),
\textcite{pekeris&jarosch58}, and \textcite{takeuchi59}.  The period of the
fundamental toroidal mode ${}_0{\rm T}_2$ was found to be 43.5 minutes,
whereas that of the ${}_0{\rm S}_2$ spheroidal mode was found to be
approximately 52 minutes; both values are a few percent smaller than modern
determinations due to an intrinsic property of the variational method---it
always yields an upper bound upon the eigenfrequency.

The first numerical integration of the radial differential equations
describing the elastic-gravitational deformation of a spherically
symmetric Earth model was performed by \textcite{takeuchi50}.
He derived the
single second-order equation governing the toroidal oscillations as well
as the three second-order equations governing the spheroidal oscillations,
and integrated the latter with the angular frequency $\om=0$ in order
to determine the degree-two static Love numbers $h$, $k$ and $l$
\index{Love number}%
of a realistic Earth; the Adams-Williamson relation
was used to obtain a consistent second-order differential
equation governing the static deformation of the fluid core.
\index{deformation!static}%
\index{static deformation}%
Takeuchi's calculation represents a significant achievement,
particularly when one recognizes that it was done without the
aid of an electronic computer.  He obtained values
$k=0.28\!-\!0.29$, $h=0.59\!-\!0.61$, $l=0.07\!-\!0.08$
that were in good agreement with a wide variety
of geophysical observations, including Darwin's measurement of
the fortnightly and monthly tides, the period of the Chandler
wobble (Love \citeyear{love09}), and the water-tube tidal tilt
measurements of \textcite{michelson&gale19}.

The modern computational era was inaugurated in a pioneering paper
by Alterman, Jarosch \& Pekeris (\citeyear{alterman&al59}).
They recast Takeuchi's radial
equations into a system of two first-order equations governing
the toroidal modes ${}_n{\rm T}_l$
and six first-order equations governing the
spheroidal modes ${}_n{\rm S}_l$,
thereby eliminating the dependence upon the
radial derivatives of the model parameters, making the results much
more amenable to numerical integration. The two first-order
equations governing the purely radial modes had been derived
earlier by \textcite{pekeris&jarosch58}.
The Runge-Kutta method was used to
calculate the eigenfrequencies and eigenfunctions of
a number of low-degree free oscillations of the Earth,
yielding periods of 44.1 and 53.7 minutes, respectively, for
the fundamental modes ${}_0{\rm T}_2$ and ${}_0{\rm S}_2$,
in excellent agreement with modern determinations.
The equations and
boundary conditions derived by Pekeris and his colleagues
are essentially the same ones that are used in spherical-Earth
normal-mode calculations today; much of modern long-period
seismology is founded upon their seminal contribution.

A major computational study of the toroidal modes of the Earth
was undertaken at approximately the same time by
\textcite{gilbert&macdonald60}; they carried out
their calculations using a spherical version of the
Thomson-Haskell layer-matrix method in lieu of numerical integration.
\index{Thomson-Haskell method}%
A large number of fundamental and higher-overtone eigenfrequencies
were computed, and Jeans' relation was used to determine the
phase and group speeds of the equivalent Love waves.
Gilbert subsequently extended the Thomson-Haskell method
to the spheroidal modes, forming the basis for the eigenfrequency
splitting calculations reported by \textcite{backus&gilbert61}.
Shortly thereafter, he abandoned the method in favor of variable-order,
variable-step Runge-Kutta integration, which could be performed
much more efficiently.  Gilbert and his collaborators went on
to dominate the field of normal-mode research during the next
two decades; his numerical integration program is the predecessor
to {\tt MINEOS\/} and {\tt OBANI\/},
\index{Mineos@\texttt{MINEOS}}%
\index{Obani@\texttt{OBANI}}%
two computer codes which are widely used
to calculate the Earth's eigenfrequencies and eigenfunctions today.

\section{Dawn of the Observational Era}
\index{free oscillations!first observation|(}%

The observational era was initiated by \textcite{benioff58},
who reported evidence for a 57-minute oscillation in the
Pasadena electromagnetic strainmeter recording of the
November 4, 1952 Kamchatka earthquake.
\index{Kamchatka 1952 earthquake}%
No Fourier analysis
of the seismogram was performed; the 57-minute period
was evidently obtained by visual inspection of a prominent
oscillation which terminates after only a few cycles.
In retrospect, it is likely that this oscillation is the
result of some sort of instrumental malfunctioning;
Benioff himself admits that ``unfortunately, the noise
level\ldots is rather high\ldots so that the measurements
are not as reliable as we would like to have them''.
A modern re-analysis of the original recording by
\textcite{kanamori76} found a number of spectral peaks
that could possibly be associated with the Earth's
normal modes; however, the amplitudes of these
oscillations were too large to have been
excited by the $M_0\approx 4\times 10^{22}$ ${\rm N}\,{\rm m}$ main shock,
calling the results into serious question.  Benioff's
purported observation was important primarily because
it stimulated further development
of long-period strainmeters and gravimeters, as well as
the ambitious computational programs of theoreticians such
as Pekeris and Gilbert.

The largest earthquake in the twentieth century
ruptured 1000 kilometers of the Nazca-America
plate boundary in Chile on May 22, 1960.
\index{Chile 1960 earthquake}%
From the standpoint
of normal-mode seismology, this $M_0\approx 2\times 10^{23}$
${\rm N}\,{\rm m}$ event could not have occurred at a more propitious
time: (1) instruments capable of recording the long-period
free oscillations had only recently been developed;
(2) computers capable of Fourier analyzing
lengthy time series had just become generally available;
(3) the technology of numerical spectral analysis had just been
codified in a widely read book by \textcite{blackman&tukey58};
and (4) the theoretical eigenfrequencies of the Earth
had just been calculated, allowing the identification
of peaks in the observed spectra.  Barely two months after
the earthquake, seismologists from the California Institute
of Technology and the University of California
at Los Angeles presented spectra at the July 1960
assembly of the International Association of Seismology
and Physics of the Earth's Interior (IASPEI)
in Helsinki which showed clear evidence for the
detection of the Earth's fundamental normal modes.
Neither of us is old enough to have been
present at this historic IASPEI meeting---indeed,
one of us was not even born---but contemporary
accounts make it sound like the Woodstock of seismology.
The Caltech group recorded the spheroidal
oscillations ${}_0{\rm S}_2$ through
${}_0{\rm S}_{38}$ and the toroidal oscillations
${}_0{\rm T}_3$ through ${}_0{\rm T}_{11}$
on quartz strainmeters located at \~{N}a\~{n}a, Peru
and Isabella, California and on pendulum
seismographs located at Pasadena, whereas the UCLA group
recorded the fundamental modes ${}_0{\rm S}_2$ through
${}_0{\rm S}_{41}$ on a Lacoste-Romberg tidal gravimeter
located at Los Angeles; the absence of any toroidal peaks
on the recording from the gravimeter aided in the mode identification.
The results of these analyses, together with an analysis of the
oscillations recorded by seismologists at the Lamont Geological
Observatory of Columbia University on their newly installed
quartz strainmeter at Ogdensburg, New Jersey and on pendulum
seismographs located at Palisades, New York, were subsequently
published in back-to-back papers in the {\em Journal of Geophysical
Research\/} (Benioff, Press \& Smith \citeyear{benioff&al61};
Ness, Harrison \& Slichter \citeyear{ness&al61};
Alsop, Sutton \& Ewing \citeyear{alsop&al61}).
Analyses of the oscillations recorded on a pendulum
located in the Grotta Gigante near Trieste, Italy
and on tiltmeters located in Paris, France were published
shortly thereafter by Bolt \& Marussi (\citeyear{bolt&marussi62})
and Connes, Blum, Jobert \& Jobert (\citeyear{connes&al62}).
The measured eigenfrequencies of the Earth were shown
by Pekeris, Alterman \& Jarosch (\citeyear{pekeris&al61a})
to agree with the theoretical
eigenfrequencies to ``better than one percent, with a distinct
preference shown for the Gutenberg (low velocity) model''.
The quest to use free-oscillation observations to
improve our knowledge of the internal structure of the
Earth had begun.

Cable dispatches were received from Caltech
and UCLA at the Helsinki meeting reporting that
both the ${}_0{\rm S}_2$ and ${}_0{\rm S}_3$ peaks
appeared to be visibly split into two.
The source of this splitting was very
quickly identified to be the rotation of the Earth,
which removes the $2l+1$ degeneracy of the
multiplets ${}_n{\rm S}_l$ and ${}_n{\rm T}_l$.
Explanatory analyses of this rotational splitting based upon
degenerate Rayleigh-Schr\"{o}dinger perturbation theory
were published by \textcite{backus&gilbert61} and
Pekeris, Alterman \& Jarosch (\citeyear{pekeris&al61b}).
\index{splitting!Coriolis}%
Unbeknownst to these workers, the effect of rotation
had previously been investigated in an astrophysical
context by \textcite{cowling&newing49} and \textcite{ledoux51}.
To lowest order, the Coriolis force gives rise to $2l+1$
equally spaced singlets with associated eigenfrequency
perturbations $\delta\om_m=m\chi\Omega$, where $m$
is the order of the complex spherical harmonic $Y_{lm}$
characterizing the zeroth-order eigenfunction,
and $\Omega$ is the angular rate of rotation;
the effect is analogous to the Zeeman
splitting of atomic spectral lines in a magnetic field.
The dimensionless splitting parameter $\chi$ is a weak
function of the Earth-model parameters
$\kappa$, $\mu$ and $\rho$ for a spheroidal multiplet,
whereas it is equal to $[l(l+1)]^{-1}$ for a toroidal multiplet.
The observed amplitudes of the singlets depend upon the
location, magnitude and geometry of the earthquake
and upon the location and polarization of the receiver.
The visibly excited peaks excited by the Chile earthquake
were shown to correspond to the split angular orders
$m=\pm 1$ in the case of the ${}_0{\rm S}_2$ multiplet and to
$m=\pm 2$ in the case of the ${}_0{\rm S}_3$ multiplet.
\index{free oscillations!first observation|)}%

\section{Spherical Earth Model Refinement}

During the two decades following the 1960 Chile earthquake,
most free-oscillation research was directed toward the
refinement of the spherically averaged structure of the
Earth.  In the beginning, the process was iterative: an
initial Earth model was used to identify the gravest
multiplets, the model was improved by trial-and-error
adjustment of $\kappa$, $\mu$ and $\rho$, enabling
identification of some of the more closely spaced
higher-frequency multiplets, and so on.
The occurrence of the $M_0\approx 8\times 10^{22}$ ${\rm N}\,{\rm m}$
Alaska earthquake
\index{Alaska 1964 earthquake}%
on March 28, 1964 supplied additional
data for this bootstrap operation (Smith \citeyear{smith66};
Slichter \citeyear{slichter67}).
Summarizing the state of progress less than ten years
after the free oscillations were first detected,
\textcite{derr69} compiled a catalogue containing
265 measured normal-mode eigenfrequencies; however,
he admitted that ``only the fundamental spheroidal
and torsional modes and a few higher spheroidal
modes\ldots can be identified with confidence''.
A number of high-$Q$ spheroidal overtone modes
were observed for the first time by Dratler, Farrell,
Block \& Gilbert (\citeyear{dratler&al71}) following
the July 31, 1970 deep-focus earthquake in Colombia.
\index{Colombia 1970 earthquake}%
Because it is situated on the exponential tail of
the associated eigenfunctions, such a deep-focus event
does not strongly excite the fundamental modes which normally
dominate the response, obscuring the low-amplitude overtones.
Attenuation filtering,
\index{attenuation filtering}%
or elimination of the first several
hours of data prior to Fourier transformation, was used to
accentuate the high-$Q$ spectral peaks.

\textcite{gilbert71a} showed that the sum
of the first-order eigenfrequency perturbations
of a multiplet ${}_n{\rm T}_l$ or ${}_n{\rm S}_l$ split by
the Earth's rotation, ellipticity and lateral heterogeneity
is identically zero, $\sum_{j=1}^{2l+1}\delta\om_j=0$,
as a consequence of the {\em diagonal sum rule\/}:
\index{diagonal sum rule}%
\begin{quote}
``We interpret this result as follows.  If the Earth
is averaged over spherical surfaces centred on its centre
of mass the result is an averaged Earth completely free
of aspherical perturbations, which we call the terrestrial
\index{terrestrial monopole}%
monopole.  The difference between the real Earth and the
terrestrial monopole is a strictly aspherical perturbation.
It is a consequence of the diagonal sum rule that the averaged
\index{diagonal sum rule}%
frequency for each multiplet, split by first-order perturbations,
belongs to the terrestrial monopole.  In other words we know that
our averaged data belong to an unperturbed spherical Earth model
that is, in fact, the spherically averaged Earth.''
\end{quote}
This provides the justification for using the measured peak
frequencies obtained from an ensemble of earthquake sources
and receivers as constraints upon the structure of the
spherically averaged Earth.
Gilbert concluded by pointing out that ``although we can hope
to obtain good coverage with receivers, we have little, if any,
control over the distribution of sources''.

The status of the
{\em mode identification problem\/}
\index{mode identification}%
was improved dramatically 
during the next few years by the clever exploitation of
data from the global, three-component World-Wide Standard
Seismographic Network (WWSSN).
\index{World-Wide Standard Seismographic Network (WWSSN)}%
The reduced
long-period sensitivity of these instruments, which were deployed
primarily to monitor underground nuclear explosions,
was offset by the availability of
large numbers of recordings.  Dziewonski \& Gilbert
(\citeyear{dziewonski&gilbert72}; \citeyear{dziewonski&gilbert73})
used 84 WWSSN recordings of the 1964 Alaska earthquake
\index{Alaska 1964 earthquake}%
to identify and measure the periods of 249 normal modes.
A number of simple discriminants, including polarization
and simple histogram analysis as well as attenuation
filtering, were used to separate and distinguish multiplets
that were closely spaced in the eigenfrequency spectrum.
An extremely important advance was made by
Mendiguren (\citeyear{mendiguren73}),
who was the first person to utilize
the capabilities of the WWSSN network as a global array.
\index{stacking}%
He developed a phase-equalization procedure in which spectra
were stacked with an appropriate change in sign in order to
accentuate the multiplet of interest and reduce the effect of
neighboring multiplets.  The method, which requires knowledge
of the source mechanism of the earthquake, was improved by
\textcite{gilbert&dziewonski75}, who applied it to 213
WWSSN recordings of the 1970 Colombia earthquake
\index{Colombia 1970 earthquake}%
and the
August 15, 1963 deep-focus event on the Peru-Bolivia border.
\index{Peru-Bolivia 1963 earthquake}%
By combining modes identified in their own and several
previous analyses, they were able to compile a standardized
data set consisting of 1064 measured eigenfrequencies of
the free oscillations of the Earth.  Thus, only fifteen years
after the first detection in 1960, approximately sixty percent
of the Earth's normal modes with periods longer than 80 seconds
had been observed and identified.

The rich harvest of high-$Q$ overtones recorded by a single
feedback accelerometer following the 1970 Colombia earthquake
provided the impetus for the development of the International
Deployment of Accelerometers (IDA)
\index{International Deployment of Accelerometers (IDA)}%
network by Agnew,
Berger, Buland, Farrell \& Gilbert (\citeyear{agnew&al76}).  The low
noise level of these instruments permitted the routine detection
of normal-mode spectral peaks following earthquakes of relatively
modest magnitude:
\vspace{1.0 mm}
\begin{center}
\begin{minipage}{3.7in}
``No longer need we wait for truly large and very infrequent
earthquakes to add to our observational knowledge of the
Earth's free oscillations.''
\end{minipage}
\end{center}
\vspace{1.0 mm}
Six stations of the fledgling IDA network recorded the long-period
oscillations excited by the $M_0=4\times 10^{21}$ ${\rm N}\,{\rm m}$
Sumbawa, Indonesia earthquake of August 19, 1977,
\index{Sumbawa 1977 earthquake}%
allowing Buland,
Berger \& Gilbert (\citeyear{buland&al79}) to extract
all five singlets of the ${}_0{\rm S}_2$ multiplet
and all seven singlets of the ${}_0{\rm S}_3$
multiplet, using a spherical-harmonic stacking procedure.
The measured spacing between the eigenfrequencies compared
favorably with updated theoretical calculations
of the effects of the Earth's rotation and hydrostatic
ellipticity performed by \textcite{dahlen&sailor79}.

Seismologists engaged in the study of terrestrial free oscillations
played a leading role in the development of a powerful new
discipline---{\em geophysical inverse theory\/}---which enabled
\index{inverse problem}%
\index{geophysical inverse theory}%
the optimal extraction of information regarding
the internal structure of the Earth from a finite
set of gross Earth data.  Models providing a best fit
to the data, subject to a variety of imposed constraints,
could be obtained by simultaneous adjustment of all the
governing parameters (Gilbert \citeyear{gilbert71b};
Jackson \citeyear{jackson72}; Wiggins \citeyear{wiggins72});
questions critical to model assessment, such as resolution and accuracy,
could also be addressed (Backus \& Gilbert \citeyear{backus&gilbert68};
\citeyear{backus&gilbert70}).  We make no attempt to
review this elegant and multi-faceted theory here;
excellent and comprehensive summaries are provided by
\textcite{menke84}, \textcite{tarantola87} and \textcite{parker94}.
The {\em Fr\'{e}chet kernels\/}
\index{Frechet kernel@Fr\'{e}chet kernel}%
relating the perturbation
$\delta\om$ in an eigenfrequency to arbitrary radial
perturbations $\delta\hspace{-0.1 mm}\kappa$,
$\delta\hspace{-0.1 mm}\mu$, $\delta\hspace{-0.2 mm}\rho$
in the incompressibility, rigidity and density were derived
using Rayleigh's principle by \textcite{backus&gilbert67}.
These kernels were used by \textcite{gilbert&dziewonski75}
to invert their 1064 measured eigenfrequencies,
together with the geodetically determined mass
and moment of inertia of the Earth; the two resulting
spherically symmetric models were whimsically dubbed
1066A and 1066B.  A small subset of radial and other
modes provided irrefutable evidence for the solidity
of the inner core (Dziewonski \& Gilbert \citeyear{dziewonski&gilbert71}).

In exhibiting the fit of models 1066A and 1066B to
body-wave travel-time data, \textcite{gilbert&dziewonski75}
added 1.6 seconds to the compressional wave observations
and 4.3 seconds to the shear-wave observations.
Such a {\em baseline correction\/}
\index{baseline correction}%
was commonly thought
to be justified by the nature of the iterative procedure
used to infer both source origin times and travel
times from measured arrival times and by the biased
siting of most seismographic stations upon continents.
A much more satisfying explanation of the discrepancy
was pointed out by Akopyan, Zharkov \& Lyubimov
(\citeyear{akopyan&al75}; \citeyear{akopyan&al76}),
\textcite{randall76}, and Liu, Anderson \& Kanamori
(\citeyear{liu&al76}): the {\em physical dispersion\/}
\index{dispersion!anelastic}%
\index{anelastic dispersion}%
that inevitably accompanies anelastic attenuation
renders the Earth's mantle slightly less rigid as
``seen'' by a typical free oscillation with a period
of 200--300 seconds than as ``seen'' by a teleseismic
shear wave with a period of 10--20 seconds.  Jeffreys
(\citeyear{jeffreys58a}; \citeyear{jeffreys58b}) and
Carpenter \& Davies (\citeyear{carpenter&davies66})
had previously sought unsuccessfully to call attention
to this effect; the latter authors investigated whether
anelastic dispersion was capable of reconciling the discrepancies
between the classical Jeffreys-Bullen and Gutenburg Earth models.
Their letter reporting the failure of the attempted
reconciliation concludes with an unambiguous warning:
\begin{quote}
`The results do, however, show that the frequency dependence
is significant and should at least be considered in any investigation
in which attenuation is an accepted part of the Earth model.
The differences between the dispersion curves\ldots ignoring or
including $Q$ are greater than the standard error obtained in
modern studies.''
\end{quote}
This admonition went unheeded for the next decade, until the
frequency dependence of the Earth's elastic parameters was
rediscovered in the context of the travel-time baseline correction.
The wide dissemination of the results of Akopyan, Zharkov
\& Lyubimov, Randall, and Liu, Anderson \& Kanamori
led to a renewed interest in the determination
of the bulk and shear quality factors $Q_{\kappa}$
and $Q_{\mu}$ within the Earth, based upon
the measured decay rates of the free oscillations
(Stein \& Geller \citeyear{stein&geller78};
Sailor \& Dziewonski \citeyear{sailor&dziewonski78};
Geller \& Stein \citeyear{geller&stein79};
Riedesel, Agnew, Berger \& Gilbert \citeyear{riedesel&al80}).

\index{Preliminary Reference Earth Model (PREM)|(}%
The Preliminary Reference Earth Model (PREM), which was
developed by \textcite{dziewonski&anderson81}
in response to a request from the International
Union of Geodesy and Geophysics for a spherically symmetric
model that could be used in geodetic and other geophysical
applications, represented the culmination of two decades of
progress in measuring and interpreting the free oscillations
of the Earth.  In addition to being anelastic and therefore
dispersive, the PREM model is transversely isotropic, with
five rather than two elastic parameters in the uppermost
mantle between 24 and 220~km depth.  More recent, higher-quality
normal-mode eigenfrequency measurements are systematically misfit
by the PREM model (Widmer \citeyear{widmer91};
Masters \& Widmer \citeyear{masters&widmer95});
in addition, a number of its features, such as the
conspicuous 220-km discontinuity, are unsupported by
body-wave reflectivity analyses (Shearer \citeyear{shearer91}).
Despite these shortcomings, we shall use PREM as the basis for most
of the numerical illustrations in this book, for
want of a more acceptable model at the time of writing.
Subsequent work has led to significantly improved models of the
radial anelastic structure of the Earth
(Masters \& Gilbert \citeyear{masters&gilbert83};
Widmer, Masters \& Gilbert \citeyear{widmer&al91};
Durek \& Ekstr\"{o}m \citeyear{durek&ekstrom96}).
Because decay-rate measurements are intrinsically much
noisier than eigenfrequency measurements, and peak widths
measured on spectral stacks are biased by splitting,
the parameters $Q_{\kappa}$ and $Q_{\mu}$
are extremely difficult to constrain.
\index{Preliminary Reference Earth Model (PREM)|)}%

\section{Source-Mechanism Determination}

Attempts were made by Alterman, Jarosch \& Pekeris
(\citeyear{alterman&al59}), Backus \& Gilbert
(\citeyear{backus&gilbert61}), and others to
calculate the amplitudes of the oscillations
excited by the 1960 Chile earthquake;
however, all of these early efforts employed
extremely simplified and, in retrospect,
unrealistic representations of the source
(explosion, point force, single couple, etc.).
The response of a spherically symmetric Earth model
to a planar fault (double couple) source was first
obtained by \textcite{saito67}.  His results were
subsequently re-derived and extended by
\textcite{gilbert70}, who noted that:
\begin{quote}
``If one can calculate the excitation of the normal
modes of the Earth due to a particular earthquake
source, one can use such calculations in an attempt
to infer the earthquake mechanism and total moment.
Some general results in normal mode theory, due to
Rayleigh and Routh about a century ago, make the
excitation calculations remarkably simple.''
\end{quote}
Gilbert and \textcite{kostrov70} independently
introduced the concept of the {\em seismic moment tensor\/}
\index{moment tensor}%
$\bM$, which is capable of representing curved
as well as planar faults, and has become the
canonical point-source model of an earthquake.
The linear dependence of the response upon the
elements of $\bM$ makes the moment-tensor
representation particularly advantageous
in source-mechanism studies, as pointed
out by \textcite{gilbert73}:
\begin{quote}
``Our problem is to determine $\bM$ given
(displacement) seismograms $\bu$ at several
locations $\br$ on the surface of the Earth.
Since $\bu$ is a linear function of $\bM$
for any given Earth model, the problem is
rather straightforward.''
\end{quote}
Mendiguren (\citeyear{mendiguren73})
and \textcite{gilbert&dziewonski75} made use of the
source-theoretical formulations of Saito and Gilbert,
respectively, in carrying out their spherical-Earth
stacking analyses of the oscillations excited by the 1970
Colombia and 1963 Peru-Bolivia deep-focus earthquakes.
Gilbert \& Dziewonski used the observed excitation amplitudes
to find the frequency-dependent moment tensors of the two events;
an alternative method more suitable for retrieving $\bM$
from a sparse network of seismographic stations was
subsequently developed by \textcite{buland&gilbert76}
and \textcite{gilbert&buland76}.

Dziewonski, Chou \& Woodhouse (\citeyear{dziewonski&al81})
extended the moment-tensor analysis of Buland \& Gilbert to allow for
a spatial and temporal shift in the centroid of the
source region:
\begin{quote}
``We describe in this report an approach that allows us to improve
the location parameters and simultaneously to modify the solution
for the moment tensor.  This, in effect, yields the `best point source'
location, which for earthquakes of finite size need not be the same as
the point of initiation of rupture.''
\end{quote}
The {\em centroid-moment tensor\/}
\index{centroid-moment tensor}%
or CMT source-mechanism
\index{CMT solution}%
determination procedure
developed by these investigators 
finds the space-time shift $\bDelta\bx$,
$\Delta t$ of the source centroid
and the moment tensor $\bM$ by fitting
long-period waveforms in the time domain; the kernels
relating the response at a given receiver to the
unknown source parameters are calculated by normal-mode summation.
The method was used in a systematic
study of 201 large and moderate earthquakes that occurred
in 1981 by \textcite{dziewonski&woodhouse83}.  This marked the
beginning of the Harvard centroid-moment tensor project,
which now routinely determines the source mechanisms
of two to three earthquakes with seismic moments greater
than $10^{16}$--$10^{17}$ ${\rm N}\,{\rm m}$ every day.
The resulting global seismicity catalogue
of more than 16,000 CMT solutions
extending from 1977 to the
present has contributed greatly
to our understanding of regional geology and tectonics,
and is one of the most valuable legacies of free-oscillation
research during the past two decades.

\section{Surface Waves}
\index{surface wave!early analyses|(}%

The spheroidal and toroidal multiplets ${}_n{\rm S}_l$ and
${}_n{\rm T}_l$ are equivalent in the limit $n\ll l$ to
propagating fundamental and higher-mode Rayleigh and Love
surface waves.  The study of elastic surface waves has
traditionally gone hand-in-hand with the study of the
free oscillations of the Earth for this reason.
The literature on surface-wave propagation, particularly in plane
stratified media, is even more voluminous than that
devoted to the elastic-gravitational normal modes;
we attempt only an extremely sketchy review here.

The existence of propagating solutions which decay
exponentially with depth beneath the surface of a
homogeneous elastic half-space was first noted by
\textcite{rayleigh85}.  The resulting surface waves are
non-dispersive.  In the case of a Poisson solid, which has
$\alpha=\sqrt{3}\hspace{0.2 mm}\beta$, where $\alpha$ and $\beta$
are the compressional and shear wave speeds, the
phase speed of Rayleigh waves is equal to
$c=0.9194\hspace{0.4 mm}\beta$.
Rayleigh conducted his investigation at a time
when seismology was still in its infancy;
however, he concluded with a remarkably prescient comment:
\begin{quote}
``It is not improbable that the surface waves
here investigated play an important role in
earthquakes\ldots.  Diverging in two dimensions
only, they must acquire at great distance from
the source a continually increasing preponderance.''
\end{quote}
\textcite{lamb04} extended Rayleigh's results,
which were limited to the free propagation of waves,
by finding the complete response of a homogeneous
half-space to both a line and point force; his classic
analysis of what we now call {\em Lamb's problem\/}
\index{Lamb's problem}%
concludes with the first synthetic seismogram ever depicted.

Inasmuch as the
first seismometers only measured horizontal
motions, the existence of prominent transverse
displacements in the ``main shock'' of an earthquake tremor,
following the ``primary'' and ``secondary'' arrivals, was one of
the earliest established facts of observational
seismology.  The particle motion
of a Rayleigh wave is a retrograde ellipse in
the vertical plane passing through the source
and receiver, and this initially impeded suggestions
that the ``main shock'' consisted of surface waves.
This controversy was resolved by
\textcite{love11}, who showed that transverse
surface waves could exist in a system consisting
of a homogeneous elastic layer overlying a
homogeneous half-space.  The resulting Love
waves are both multi-modal and dispersive,
with a phase speed $c_n(\om)$ that depends
upon the overtone number $n$ and the frequency
$\om$.  With remarkable foresight, Love
argued that surface-wave dispersion was responsible
for the observed oscillatory character of the ``main shock'':
\begin{quote}
``The explanation which I wish to suggest is that the
oscillations are due to dispersion\ldots.  The actual
motion may, of course, be analysed into\ldots an
aggregate of simple harmonic wave-trains, each
travelling with the wave-velocity appropriate
to its wave-length.''
\end{quote}
The dispersion relation for Rayleigh waves in a system
consisting of a surficial layer overlying a half-space
is considerably more difficult to obtain.  Love treated
the case in which both the layer and the half-space are
incompressible, \textcite{bromwich98} considered an incompressible
fluid layer overlying an elastic half-space, and \textcite{stoneley26a}
extended his analysis to allow for the compressibility
of the fluid.  The general case of a solid elastic layer overlying
an elastic half-space was finally solved by \textcite{sezawa27}
and \textcite{stoneley28}.  The non-gravitating version of Rayleigh's
variational principle governing propagation in an arbitrarily
stratified half-space was first given for Love waves by
\textcite{meissner26} and for Rayleigh waves by \textcite{jeffreys35}. 

A number of early investigators, including Angenheister
(\citeyear{angenheister21}) and Tams (\citeyear{tams21}),
noticed that surface waves propagate at different speeds
over oceanic and continental paths.  The first systematic
study of this phenomenon, which distinguished clearly between
waves recorded on the longitudinal and transverse components,
and which sought to use Love's theoretical results for the
latter to determine the thickness of the oceanic and continental
crust, was made by \textcite{gutenberg24}.  To measure the speed of
Love waves in the period range 10--60 seconds, Gutenberg divided the
epicentral distance by the arrival time.  \textcite{stoneley25} pointed
out that such measurements corresponded to the theoretical group speed $C$
\index{group speed}%
\index{speed!group}%
rather than the phase speed $c$, as Gutenberg had assumed;
the two speeds are related by $C=c+k(dc/\hspace{-0.2 mm}dk)$,
where $k$ is the wavenumber.  Other attempts to constrain crustal
structure by comparing measured and theoretical group speeds followed;
\textcite{stoneley&tillotson28} considered the effect of a two-layer
crust---a ``granitic'' upper layer overlying a ``dioritic'' lower
layer---upon Love-wave propagation.  Most of these early efforts
were relatively inconclusive due to the scatter in the measurements
and the paucity of quantitative theoretical results for pertinent
crustal models.  The gross thickness and character
of the continental crust were
inferred primarily by the analysis of Pn, Sn and other body waves
from near-focus earthquakes, whereas the structure of the
much thinner oceanic crust was
ultimately determined by shipboard seismic refraction experiments.

The theoretical understanding of dispersive surface-wave propagation
was advanced considerably by the work on underwater sound transmission
conducted during World War II by \textcite{pekeris48} and his collaborators.
Pekeris clarified the role of the group speed in controlling the character
of a surface-wave seismogram, and coined the term {\em Airy phase\/} to
\index{Airy phase}%
describe the large-amplitude signal associated with a group-speed maximum
or minimum.  This improved understanding was exploited by seismologists
during the post-war years, with the result that reliable group speeds of
fundamental Rayleigh and, to a lesser extent, Love waves in both continental
and oceanic regions had soon been determined.  \textcite{ewing&press54} used
Benioff's recording of the 1952 Kamchatka earthquake
\index{Kamchatka 1952 earthquake}%
at Pasadena to extend
the dispersion curve for Rayleigh waves up to a period of 480 seconds;
the multi-orbit wavegroups R6 through R15 could be identified, the latter
having undergone more than seven complete circumnavigations of the Earth.
The status of post-war surface-wave research was summarized in a
major monograph by Ewing, Jardetzky \& Press (\citeyear{ewing&al57});
their description of the dispersive characteristics
of mantle Rayleigh waves---``a minimum value of group velocity of 3.5~km/sec
at a period of 225 sec, a short-period limit of 3.8~km/sec at 70 sec, and the
flattening of the curve for periods greater than 400 sec''---remains
accurate to this day.
Mantle Love waves on the other hand exhibit a nearly constant group speed
of approximately 4.4 kilometers per second in the period range 70--400 seconds,
resulting in an impulsive arrival known as a {\em G wave\/} after
\index{G wave}%
Gutenberg,
who first called attention to it
(Gutenberg \& Richter \citeyear{gutenberg&richter34}).

Early measurements of regional surface-wave phase speed employed either
a three-station triangulation technique
(Press \citeyear{press56}) or a two-station
technique which relied upon a suitable geographical alignment of the source
and receivers.  The two-station method, which was introduced but not applied
to the analysis of any earthquake data by \textcite{sato55}, was used in a
classic study of the Canadian shield by \textcite{brune&dorman63}.
The phase speeds of long-period mantle waves were determined by means
of a {\em great-circle\/} technique, utilizing multiple passages of the
same wavegroup through a single station.  \textcite{sato58} and
\textcite{nafe&brune60} made the first such measurements for Love
and Rayleigh waves, respectively, using G3--G1 and G4--G2 arrivals from
the February 1, 1938 New Guinea earthquake,
\index{New Guinea 1938 earthquake}%
G3--G1 arrivals from the
1952 Kamchatka earthquake, and R5--R3 arrivals from the
August 15, 1950 Assam earthquake
\index{Assam 1950 earthquake}%
recorded at Pasadena.
Brune, Nafe \& Alsop (\citeyear{brune&al61}) subsequently showed
that these pioneering measurements were slightly too high due to
the neglect of the $\pi\hspace{-0.3 mm}/\hspace{-0.1 mm}2$
{\em polar phase shift\/}
\index{polar phase shift}%
\index{phase shift!polar}%
at the source and antipodal caustics.
\index{caustic}%

A technique for measuring regional surface-wave phase
speeds using only minor-arc data
was introduced by Brune, Nafe \& Oliver (\citeyear{brune&al60});
however, this {\em single-station\/} method,
\index{single-station method}%
which requires knowledge of the initial phase
of the waves leaving the source, did not become
reliable until the double-couple excitation problem
was solved, initially by \textcite{haskell64} and
\textcite{ben-menahem&harkrider64}, and then in a more useful
form by \textcite{saito67}.  Recordings of the R1 and G1
minor-arc surface waves from seventeen
earthquakes with known focal mechanisms were used in a major study
of the structure beneath the eastern Pacific Ocean by \textcite{forsyth75}.
The reliability of modern earthquake source-mechanism determinations and
the availability of the Harvard CMT catalogue make this the preferred
phase-speed measurement method today.

The isolation of higher-mode Love and Rayleigh waves is considerably
more difficult than that of fundamental-mode waves due to the near
coincidence of the group speeds near 4.4~km/s for periods lower than
100 seconds.  The earliest analyses focused upon group-speed measurements
of shorter-period crustal-guided phases such as Lg and Rg
(Press \& Ewing \citeyear{press&ewing52}).  
\textcite{nolet77} and \textcite{cara78} used
an innovative multi-station stacking technique
to measure the phase speeds of higher-mode Rayleigh waves
propagating across western Europe and North America, respectively;
they obtained results up to overtone number $n=6$ and phase
speed $c=7.5$~km/s in the period range between 25 and 100 seconds.
The Fr\'{e}chet kernels
\index{Frechet kernel@Fr\'{e}chet kernel}%
relating the change $\delta c$
in the phase speed of a fundamental or higher-mode
surface wave to changes $\delta\hspace{-0.1 mm}\kappa$,
$\delta\hspace{-0.1 mm}\mu$, $\delta\hspace{-0.2 mm}\rho$
in the model parameters were first obtained using Rayleigh's
principle for a stratified half-space by
Takeuchi, Dorman \& Saito (\citeyear{takeuchi&al64}).
Successively higher modes of a given period ``feel''
more deeply into the Earth, thereby offering
greater resolution at depth.

\index{attenuation!first measurement|(}%
\index{surface-wave attenuation!first measurement|(}%
The first measurement of the {\em attenuation\/} of seismic waves
within the Earth was made by Angenheister (\citeyear{angenheister06}).
He compared the amplitudes of 15--20 second surface waves propagating
along the minor and major arcs to G\"{o}ttingen from five distant earthquakes.
It is not clear from his discussion whether
the measurements pertain to Love waves or to
Rayleigh waves; he refers to the minor-arc and major-arc arrivals as
${\rm W}_1$ and ${\rm W}_2$, respectively.  Expressing the amplitude
ratio as $A_2/\hspace{-0.2 mm}A_1=e^{-\gamma d}$,
where $d$ is the differential distance
in kilometers, he found the attenuation coefficient to be
$\gamma = 1.8\times 10^{-4}$ to $\gamma = 3.4\times 10^{-4}$~${\rm km}^{-1}$.
The corresponding quality factor for 20-second waves is $Q=150$--300.
This pioneering measurement of the Earth's anelastic attenuation is a
noteworthy accomplishment, in view of the extremely primitive understanding
of surface-wave propagation at the time.
\index{attenuation!first measurement|)}%
\index{surface-wave attenuation!first measurement|)}%
\index{surface wave!early analyses|)}%

\section{Lateral Heterogeneity}
\index{lateral heterogeneity!early analyses|(}%

The observed regional variation in surface-wave dispersion at
periods below 100 seconds is the result of the strong contrast
in continental and oceanic crustal thickness and other near-surface
structural differences; this was clearly recognized by the post-war
seismologists who sought to utilize short-period surface waves
as a tool for delineating regional variations
in crustal structure.  The first evidence for lateral variations
$\delta\hspace{-0.2 mm}\beta$ in the shear-wave speed of the
{\em upper mantle\/} was provided by \textcite{toksoz&ben-menahem63};
they measured the phase speeds of
50--400 second Love waves over six great-circle paths, finding
differences in excess of one percent, ``much greater than the
experimental error''.  The inverse problem of inferring upper-mantle
heterogeneity from phase-speed measurements was investigated by
\textcite{backus64}, who pointed out that only the even-degree part
of the Earth could be determined from great-circular average data.
Regionalization schemes which divided the surface of the Earth into
``shield'', ``oceanic'', and ``tectonic'' provinces were introduced
by \textcite{toksoz&anderson66} and \textcite{kanamori70} in an
attempt to overcome this ambiguity.

\index{peak shifts|(}%
Variations in the apparent central frequencies of fundamental-mode
multiplets ${}_0{\rm T}_l$ or ${}_0{\rm S}_l$ provide an additional
constraint upon upper-mantle lateral heterogeneity; such shifts in the
location of the spectral peaks are the result of interference
among the unresolvably split singlets comprising each multiplet.
\textcite{jordan78} and \textcite{dahlen79} used degenerate
normal-mode perturbation theory to show that such peak-shift
measurements depend only upon the structure underlying the
source-receiver great-circle path, in the limit $s_{\rm max}\ll l$,
where $s_{\rm max}$ is the maximum spherical-harmonic degree
of the lateral heterogeneity and $l$ is the degree of the
multiplet under consideration; normal-mode peak frequencies
and great-circular surface-wave phase speeds may be regarded
as interchangeable data in this geometrical-optics limit.
\textcite{silver&jordan81} analyzed 72 accelerograms from the
IDA network to obtain 2193 apparent central frequencies for the
fundamental spheroidal modes ${}_0{\rm S}_5$ through ${}_0{\rm S}_{43}$,
and sought to interpret these using a more sophisticated
regionalization scheme comprised of three oceanic provinces
sorted by age and three continental provinces ``divided
accorded to their generalized tectonic history during the
Phanerozoic''.

A larger data set consisting of 3934 reliable
measurements from 557 IDA recordings was subsequently collected
by Masters, Jordan, Silver \& Gilbert (\citeyear{masters&al82}).
Upon plotting their observed peak shifts at the poles of the source-receiver
great circles, they discovered a remarkably coherent pattern,
which led them to eschew a regionalization approach in favor
of a spherical-harmonic representation of the heterogeneity:
\begin{quote}
``Free-oscillation data reveal heterogeneity in the Earth's
mantle whose geographical pattern is dominated by spherical harmonics
of angular degree two\ldots The heterogeneity can be modelled
as localized in the transition zone (420--670~km depth) and may
be related to a large-scale component of mantle convection.''
\end{quote}
The variance reduction achieved by their simple degree-two
 model was a remarkable seventy percent; spherical-harmonic
representations of the three-dimensional structure of the
Earth have been employed in the majority of global tomographic
studies since that time.
\index{peak shifts|)}%

\index{Woodhouse-Dziewonski method|(}%
\textcite{woodhouse&dziewonski84} devised a procedure for the
inversion of seismic waveform data in the time domain and applied
it to a data set consisting of 2000 IDA and Global Digital Seismographic
Network (GDSN) recordings of the mantle waves generated by
53 earthquakes ranging in moment from $4\times 10^{18}$ to
$3.6\times 10^{21}$ ${\rm N}\,{\rm m}$.
The corresponding synthetic seismograms were calculated
using a {\em path-average\/} or {\em great-circle approximation\/}, which accounts
\index{great-circle approximation}%
for the effect of the Earth's lateral heterogeneity by slight
modifications to the eigenfrequency and apparent epicentral
distance associated with each multiplet in a conventional
mode-summation algorithm; the method assumes that the response
depends only upon the laterally averaged structure beneath the
source-receiver great-circle path, but it is sensitive to odd
as well as even spherical-harmonic degrees because it treats
the various arrivals R1, R2,\hspace{0.2mm}\ldots or G1, G2,\hspace{0.2mm}\ldots separately.
Woodhouse \& Dziewonski's model of shear-wave heterogeneity
$\delta\hspace{-0.2 mm}\beta$ in the upper mantle, ``expanded up
to degree and order 8 in spherical harmonics, and described by
a cubic polynomial in depth for the upper 670~km'', represented
a significant increase in both
lateral and radial resolution.  Despite the absence
of any a priori tectonic regionalization, their waveform inversion
study showed that ``shields and ridges are major features in the
depth interval 25--250~km''.  
\index{Woodhouse-Dziewonski method|)}%

Global tomographic studies since 1985 have sought not only to improve the
resolution and reproducibility of images of the Earth's three-dimensional
elastic and anelastic structure, but, more importantly, to address
fundamental questions regarding the geodynamical and compositional
causes of the heterogeneity.  Modern analyses make use of a wide
variety of seismological data, including complete spectra of strongly
split free-oscillation multiplets, complete waveforms of both body
and surface waves, measured phase speeds of R1, R2,\hspace{0.2mm}\ldots or
G1, G2,\hspace{0.2mm}\ldots waves, measured absolute P, PKP and PKIKP
and differential SS--S and ScS--S travel times, and observations
of upper-mantle conversions, reflections and reverberations.
We do not discuss many of these procedures, nor do we attempt
to review or critically appraise the fidelity of the current
generation of three-dimensional mantle models, inasmuch as our
emphasis is upon the free oscillations of the Earth and related
mode-summation and surface-wave ray-theoretical methods of
calculating synthetic long-period seismograms.
Summary accounts devoted to the broader topic of global seismic
tomography are provided by \textcite{romanowicz91} and
\textcite{ritzwoller&lavely95}.
\index{lateral heterogeneity!early analyses|)}%
